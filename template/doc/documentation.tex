% ensure that space is added after \latex
\let\oldlatex\latex
\renewcommand{\latex}{\oldlatex\xspace}

% add package to index, when printed
\let\OrigDemoPackage\demopackage
\renewcommand{\demopackage}[1]{%
\OrigDemoPackage{#1}%
\index{package!#1}%
}%

% add command to index, when printed
\let\OrigDemocs\democs
\renewcommand{\democs}[1]{%
\OrigDemocs{#1}%
\index{command!\bs{}#1}%
}%

% -------------------------------------------------
% \printCodeFromFile
% -------------------------------------------------
\newcounter{lstLastPage}
\newcounter{lstLastLine}
\setcounter{lstLastLine}{0}
\setcounter{lstLastPage}{0}
%
\newcommand{\printCodeFromFile}[3][]{%
\ifstrempty{#1}{}{%
  \setcounter{lstLastPage}{#1}%
}%
\setcounter{lstLastLine}{#2}%
%
\lstinputlisting[%
  firstnumber=\thelstLastPage,%
  firstline=\thelstLastPage,%
  lastline=\thelstLastLine,%
  nolol=true,
  style=lstStyleLaTeX]%
  {#3}%
%
\setcounter{lstLastPage}{#2}
\addtocounter{lstLastPage}{1}
}

% -------------------------------------------------
% \file and \labelfile
% -------------------------------------------------
% Code copied from http://tex.stackexchange.com/
%  questions/65639/how-to-create-my-on-ref-label-system/
% with later modifications.
% Thanks to Heiko Oberdiek for providing this answer !
% -------------------------------------------------
\DeclareUrlCommand{\FileName}{\urlstyle{tt}}

%%%
\newcounter{file}
% hyperref uses \theH<counter>
\providecommand*{\theHfile}{\thefile}

\newcommand*{\labelfile}[1]{%
  \renewcommand*{\theHfile}{#1}%
  \refstepcounter{file}%
  \phantomsection
  \label{file:#1}%
  \index{files!#1@\string\FileName{#1}}%
}
%%%

\newcommand*{\file}[1]{%
  \hyperref[file:#1]{\FileName{#1}}%
} 


% =========================================================================
\chapter{Overview}

% =========================================================================
\chapter{Main file (LaTeXTemplate.tex)}
\labelfile{LaTeXTemplate.tex}

% -------------------------------------------------------------------------
\section{Code before the documentclass}

% ~~~~~~~~~~~~~~~~~~~~~~~~~~~~~~~~~~~~~~~~~~~~~~~~~~~~~~~~~~~~~~~~~~~~~~~~~
\subsection{magic shortcodes}
\printCodeFromFile{1}{LaTeXTemplate.tex}

% ~~~~~~~~~~~~~~~~~~~~~~~~~~~~~~~~~~~~~~~~~~~~~~~~~~~~~~~~~~~~~~~~~~~~~~~~~
\subsection{bug fix packages}
\printCodeFromFile[3]{6}{LaTeXTemplate.tex}

% -------------------------------------------------------------------------
\section{Documentclass}
In this template only classes from Koma-Script (Version 3) can be used.
Other classes would result in a non compiling template and are not supported therefore. 

In document class options some of the most important settings for the document are configured, such as paper size, font size and language of the document.
\printCodeFromFile{22}{LaTeXTemplate.tex}

% -------------------------------------------------------------------------
\section{Preamble (packages and settings)}
The code after \texttt{documentclass} and before the document starts is called preamble. All functionality and layout is loaded and configured there. The following sections show in which order things are loaded and configured.

% ~~~~~~~~~~~~~~~~~~~~~~~~~~~~~~~~~~~~~~~~~~~~~~~~~~~~~~~~~~~~~~~~~~~~~~~~~
\subsection{Packages that come first}
The following code loaded all packages that must be loaded before anything else. In this template this is necessary for all packages that are using by the template itself in control sequences.
\printCodeFromFile[27]{29}{LaTeXTemplate.tex}

% ~~~~~~~~~~~~~~~~~~~~~~~~~~~~~~~~~~~~~~~~~~~~~~~~~~~~~~~~~~~~~~~~~~~~~~~~~
\subsection{Encoding}
Selection of encoding of the LaTeX files and the encoding of the file system. The latter is primarily depended on the operating system.
\printCodeFromFile[35]{46}{LaTeXTemplate.tex}

% ~~~~~~~~~~~~~~~~~~~~~~~~~~~~~~~~~~~~~~~~~~~~~~~~~~~~~~~~~~~~~~~~~~~~~~~~~
\subsection{Packages, layout, fonts and custom commands}
Selection of fonts, packages (functionality), the style (layout) and custom commands that are required by the template.

TODO: add links to subfiles
\printCodeFromFile[52]{60}{LaTeXTemplate.tex}

% ~~~~~~~~~~~~~~~~~~~~~~~~~~~~~~~~~~~~~~~~~~~~~~~~~~~~~~~~~~~~~~~~~~~~~~~~~
\subsection{Configuration}
All the configuration code shown here is separated from the files \file{preamble/packages.tex} or \file{preamble/style.tex} because they are either system or target specific.
\medskip\\\noindent
%
Selection of link colors: The links can either be displayed in colors for a pdf document or be displayed in black for a print document.
\printCodeFromFile[69]{73}{LaTeXTemplate.tex}
%
Here possible options are selectable, which configure the way the pdf document is opened.
\printCodeFromFile[74]{87}{LaTeXTemplate.tex}
%
The backend and encodings for biblatex are configured in the following code.
\printCodeFromFile[89]{97}{LaTeXTemplate.tex}

% ~~~~~~~~~~~~~~~~~~~~~~~~~~~~~~~~~~~~~~~~~~~~~~~~~~~~~~~~~~~~~~~~~~~~~~~~~
\subsection{Custom definitions}
With the following files custom macros (\file{macros/newcommands.tex} and additional hypernation patterns \file{premabel/Hyphenation.tex} are loaded. 
\printCodeFromFile[103]{106}{LaTeXTemplate.tex}

% ~~~~~~~~~~~~~~~~~~~~~~~~~~~~~~~~~~~~~~~~~~~~~~~~~~~~~~~~~~~~~~~~~~~~~~~~~
\subsection{Execution of commands}
\label{sec:ExecutionOfCommands}
Several packages only work if their make-commands are executed. Examples are index, glossaries and such. Here these are grouped in the file \file{macros/makeCommands.tex}. 

\democs{listfiles} tells \latex to print all files loaded during compilation in a file list at the end of the log-file.
%
\printCodeFromFile[113]{115}{LaTeXTemplate.tex}

% ~~~~~~~~~~~~~~~~~~~~~~~~~~~~~~~~~~~~~~~~~~~~~~~~~~~~~~~~~~~~~~~~~~~~~~~~~
\subsection{Bibliography data}
With biblatex the bibliography files are loaded before the document starts. 
They are loaded with the command \democs{addbibresource} and the file is included without the \texttt{.bib} file extension. Multiple files bibliography files are added with multiple \democs{addbibresource} commands.
\printCodeFromFile[121]{129}{LaTeXTemplate.tex}

% -------------------------------------------------------------------------
\section{The document (the content)}
It start with \texttt{\bs{}begin\{document\}} and ends with \texttt{\bs{}end\{document\}}.
The code in-between includes all the content for the document. Nevertheless the code is filled with necessary style and settings commands.
\printCodeFromFile[145]{147}{LaTeXTemplate.tex}

% ~~~~~~~~~~~~~~~~~~~~~~~~~~~~~~~~~~~~~~~~~~~~~~~~~~~~~~~~~~~~~~~~~~~~~~~~~
\subsection{Acronyms}
If you want to use acronyms you can fill them in the file loaded here:
\printCodeFromFile[149]{150}{LaTeXTemplate.tex}
%
% ~~~~~~~~~~~~~~~~~~~~~~~~~~~~~~~~~~~~~~~~~~~~~~~~~~~~~~~~~~~~~~~~~~~~~~~~~
\subsection{Title page}
The page style and the page numbering for the title page is set up with this code
\printCodeFromFile[152]{154}{LaTeXTemplate.tex}
%
followed by the title page in file \file{content/title}.
\printCodeFromFile[157]{158}{LaTeXTemplate.tex}

% ~~~~~~~~~~~~~~~~~~~~~~~~~~~~~~~~~~~~~~~~~~~~~~~~~~~~~~~~~~~~~~~~~~~~~~~~~
\subsection{Abstract}
An abstract is common in phd thesis, but unusual in master and bachelor thesis. If you do not require an abstract just comment out the following lines.
\printCodeFromFile[162]{163}{LaTeXTemplate.tex}

% ~~~~~~~~~~~~~~~~~~~~~~~~~~~~~~~~~~~~~~~~~~~~~~~~~~~~~~~~~~~~~~~~~~~~~~~~~
\subsection{Declaration}
These lines load the document \file{content/Z-Declaration} in which you can state that the whole document is based on your ideas and written by only yourself. As far as I know this is required in bachelor and master thesis, but not part of phd-thesis. Comment out this line if you do not require it.
\printCodeFromFile[165]{166}{LaTeXTemplate.tex}

% ~~~~~~~~~~~~~~~~~~~~~~~~~~~~~~~~~~~~~~~~~~~~~~~~~~~~~~~~~~~~~~~~~~~~~~~~~
\subsection{Frontmatter}
The front pages of a thesis typically contain the table of contents followed by other lists. Here these are the symbol list, an acronym list and a glossary.

These lines only setup the page style and the line numbering for the front pages. The first line sets up as pages with headings defined by \texttt{srcheadings} and the line numbering is applied by the command \democs{frontmatter} in the second line.
\printCodeFromFile[168]{169}{LaTeXTemplate.tex}

% ~~~~~~~~~~~~~~~~~~~~~~~~~~~~~~~~~~~~~~~~~~~~~~~~~~~~~~~~~~~~~~~~~~~~~~~~~
\subsection{Table of contents}
The table of contents is inserted with \democs{tableofcontents}. Additionally it is added to the pdf bookmarks.
\printCodeFromFile[173]{175}{LaTeXTemplate.tex}

% ~~~~~~~~~~~~~~~~~~~~~~~~~~~~~~~~~~~~~~~~~~~~~~~~~~~~~~~~~~~~~~~~~~~~~~~~~
\subsection{Lists: acronym, symbols, glossaries}
These are loaded if the package for all these lists is loaded and the standard style, which requires the \demopackage{longtable} package is loaded. If you do not require all these lists comment those out that you do not want. The make commands required for building these lists were already executed, see \vref{sec:ExecutionOfCommands}. The styles of these lists are defined in file \file{preambel/style-glossaries.tex}.
\printCodeFromFile[177]{187}{LaTeXTemplate.tex}

% ~~~~~~~~~~~~~~~~~~~~~~~~~~~~~~~~~~~~~~~~~~~~~~~~~~~~~~~~~~~~~~~~~~~~~~~~~
\subsection{Main Document}
This is the part which contains all the content. It start with \democs{mainmatter}, which sets up the line numbering and is followed by a list of files loaded with \democs{include}. The latter is important to ensure that \democs{includeonly} works.
\printCodeFromFile[189]{200}{LaTeXTemplate.tex}

% ~~~~~~~~~~~~~~~~~~~~~~~~~~~~~~~~~~~~~~~~~~~~~~~~~~~~~~~~~~~~~~~~~~~~~~~~~
\subsection{Bibliography}
The bibliography is placed directly after the main content. It however must not be placed in the appendix. The layout of the bibliography is defined in file \file{preambel/style-biblatex.tex}.
%
\printCodeFromFile{207}{LaTeXTemplate.tex}

% ~~~~~~~~~~~~~~~~~~~~~~~~~~~~~~~~~~~~~~~~~~~~~~~~~~~~~~~~~~~~~~~~~~~~~~~~~
\subsection{Lists of figures, tables, listings}
Several lists are automatically created by \latex. The most common are the list of figures and list of tables. If one of these lists is not required the responsible line can be commented out. 
%
\printCodeFromFile{213}{LaTeXTemplate.tex}

% ~~~~~~~~~~~~~~~~~~~~~~~~~~~~~~~~~~~~~~~~~~~~~~~~~~~~~~~~~~~~~~~~~~~~~~~~~
\subsection{Lists of listings}
The list of listings is one of the additional lists that can be created. 
It can only be created if the total number of list stays below the total number of possible file outputs. For more information see section \ref{sec:problems:write}.

\printCodeFromFile[214]{217}{LaTeXTemplate.tex}

% ~~~~~~~~~~~~~~~~~~~~~~~~~~~~~~~~~~~~~~~~~~~~~~~~~~~~~~~~~~~~~~~~~~~~~~~~~
\subsection{Appendix}
The appendix contains contains additional information that do not fit into the main text of the thesis and must contain only information which is \emph{not} necessary for the understanding of the main text. Therefore the appendix is not treated as part of the thesis in the valuation.

The appendix is started with \democs{appendix} and manually added to the table of contents. In the last line the file \file{content/Z-Appendix.tex} is loaded which contains all further chapters and sections of the appendix.
%
\printCodeFromFile[218]{224}{LaTeXTemplate.tex}

% ~~~~~~~~~~~~~~~~~~~~~~~~~~~~~~~~~~~~~~~~~~~~~~~~~~~~~~~~~~~~~~~~~~~~~~~~~
\subsection{Publications and Curriculum Vita}
The list of publications is loaded with file \file{content/Z-Publications.tex} and the cv with \file{content/Z-CV.tex}. These files should only be loaded in case of a phd-thesis. For bachelor and master thesis these lines should be commented out.
%
\printCodeFromFile{229}{LaTeXTemplate.tex}

% ~~~~~~~~~~~~~~~~~~~~~~~~~~~~~~~~~~~~~~~~~~~~~~~~~~~~~~~~~~~~~~~~~~~~~~~~~
\subsection{Index}
An index is very useful for finding a topic in a large document. It is however also very time consuming to create a good index. If you are not sure that your index content is worth to include it in your thesis you should comment these lines out.

The setup for the index is done in file \file{preambel/style-index.tex}.
\printCodeFromFile{234}{LaTeXTemplate.tex}

% ~~~~~~~~~~~~~~~~~~~~~~~~~~~~~~~~~~~~~~~~~~~~~~~~~~~~~~~~~~~~~~~~~~~~~~~~~
\subsection{Thanks}
It is common to add a page at the end of the document where the author thanks all people who helped in the creation of the thesis. 
\printCodeFromFile{239}{LaTeXTemplate.tex}

% ~~~~~~~~~~~~~~~~~~~~~~~~~~~~~~~~~~~~~~~~~~~~~~~~~~~~~~~~~~~~~~~~~~~~~~~~~
\subsection{End}
Finally the main file is closed with 
\printCodeFromFile[240]{245}{LaTeXTemplate.tex}

Any content after this line will not be executed.

% =========================================================================
\chapter{Preamble files}

% -------------------------------------------------------------------------
\section{fonts/fonts.tex}
\labelfile{fonts/fonts.tex}

This file loads the packages \demopackage{cmap}, \demopackage{fontenc} and
\demopackage{textcomp}. The default font is \emph{Latin Modern}, loaded with package \demopackage{lmodern}. Further font families and typical font combinations 
are presented but not loaded.

\printCodeFromFile[1]{96}{fonts/fonts.tex}

% ~~~~~~~~~~~~~~~~~~~~~~~~~~~~~~~~~~~~~~~~~~~~~~~~~~~~~~~~~~~~~~~~~~~~~~~~~
\subsection{fonts/fonts-lmodern-sansmath.tex}
\labelfile{fonts/fonts-lmodern-sansmath.tex}

This file defines a sans math version for package \demopackage{lmodern}. 
It is activated with \democs{mathversion\{sans\}}.

\printCodeFromFile[1]{12}{fonts/fonts-lmodern-sansmath.tex}

% ~~~~~~~~~~~~~~~~~~~~~~~~~~~~~~~~~~~~~~~~~~~~~~~~~~~~~~~~~~~~~~~~~~~~~~~~~
\subsection{fonts/fonts-commercial.tex}
\labelfile{fonts/fonts-commercial.tex}
If you own commercial fonts and have the required \latex packages installed then this file might be of interest for you. It shows how to load \emph{some} of the available fonts for pdflatex. The file \file{fonts/fonts.tex} must still be loaded because it contains further packages that are required.

For MyriadPro and MinionPro the code is extracted into extra files % (\file{fonts/fonts-MinionPro.tex} and \file{fonts/fonts-MyriadPro.tex})
because these package come with a lot of functionality and thus options.

\printCodeFromFile[1]{100}{fonts/fonts-lmodern-sansmath.tex}

% ~~~~~~~~~~~~~~~~~~~~~~~~~~~~~~~~~~~~~~~~~~~~~~~~~~~~~~~~~~~~~~~~~~~~~~~~~
\subsubsection{fonts/fonts-MinionPro.tex}
\labelfile{fonts/fonts-MinionPro.tex}
File that loads MinionPro and takes care of the package loaded order.

\printCodeFromFile[1]{33}{fonts/fonts-MinionPro.tex}

% ~~~~~~~~~~~~~~~~~~~~~~~~~~~~~~~~~~~~~~~~~~~~~~~~~~~~~~~~~~~~~~~~~~~~~~~~~
\subsubsection{fonts/fonts-MyriadPro.tex}
\labelfile{fonts/fonts-MyriadPro.tex}
File that loads MyriadPro and takes care of the package loaded order. MyriadPro must be loaded after MinionPro if both shall be loaded.

\printCodeFromFile[1]{35}{fonts/fonts-MyriadPro.tex}

% -------------------------------------------------------------------------
\section{preambel/packages.tex}
\labelfile{preambel/packages.tex}

% ~~~~~~~~~~~~~~~~~~~~~~~~~~~~~~~~~~~~~~~~~~~~~~~~~~~~~~~~~~~~~~~~~~~~~~~~~
\subsection{Package sections}

This is the file that loads all packages. The packages are grouped together according to there usage. However in many cases the loading order must be different. Therefore the loading order is corrected by commands such as \democs{ExecuteAfterPackage}. If packages can only be loaded after other packages have been loaded or must not be loaded in a special combination this is recognized and the package either loaded or not in order to prevent the template from not compiling. 

All package groups, named within this text \emph{sections}, start with the command \democs{BeginTemplateSection} and end with \democs{EndTemplateSection}. If these section are included in the compilation or excludes (not compiled) is defined at the beginning of the file:

\printCodeFromFile[1]{24}{preambel/packages.tex}

If you do not require all sections in your document you can thus change the setting from \emph{true} to \emph{false} in all section definitions you do not want to include in the compilation.

The whole template should compile with any section excluded except section \emph{PackagesBase}. If this is not the case please submit a bug report.

% ~~~~~~~~~~~~~~~~~~~~~~~~~~~~~~~~~~~~~~~~~~~~~~~~~~~~~~~~~~~~~~~~~~~~~~~~~
\subsection{Base packages}
\label{sec:packages:base}
The following packages provide basic functionality such as language selections, graphics and colors. Since most other packages require these to be loaded they are loaded here at the beginning. 

\begin{itemize}[noitemsep]
\item \demopackage{calc}
\item \demopackage{babel}, \demopackage{translater}
\item \demopackage{xcolor}
\item \demopackage{graphicx}
\item \demopackage{epstopdf}
\item \demopackage{ragged2e}
\end{itemize}

The application of each package is given with a short description in the source code. The documentation file name and package loading order requirements are also included in the source code. 

\printCodeFromFile[25]{76}{preambel/packages.tex}

% ~~~~~~~~~~~~~~~~~~~~~~~~~~~~~~~~~~~~~~~~~~~~~~~~~~~~~~~~~~~~~~~~~~~~~~~~~
\subsection{Bug fixing packages}

\TeX{} may be bug-free, but \latex and its packages are certainly not free of bugs.  Most packages are updated in short term if bugs are encountered. \latex however has the philosophy to maintain a document setting stability. Therefore bugs in the base \latex system are not fixed, even if they are well known. However, some of them are fixed by extension packages. Others are fixed by special packages, which are loaded here.

\begin{itemize}[noitemsep]
\item \demopackage{fixltx2e}
\item \demopackage{marginnote}, (\demopackage{mparhack})
\item \demopackage{scrhack}
\item \demopackage{marginfix}
\item \demopackage{xspace}
\end{itemize}


\printCodeFromFile[77]{112}{preambel/packages.tex}

% ~~~~~~~~~~~~~~~~~~~~~~~~~~~~~~~~~~~~~~~~~~~~~~~~~~~~~~~~~~~~~~~~~~~~~~~~~
\subsection{Font packages}
This section is rather empty since the fonts and most of the related packages are already loaded in the file \file{fonts/fonts.tex}.

\begin{itemize}[noitemsep]
\item \demopackage{relsize}
\end{itemize}

\printCodeFromFile[113]{123}{preambel/packages.tex}

% ~~~~~~~~~~~~~~~~~~~~~~~~~~~~~~~~~~~~~~~~~~~~~~~~~~~~~~~~~~~~~~~~~~~~~~~~~
\subsection{Math packages}
The base package for all math in \latex is the package \demopackage{amsmath}. The other packages are not necessary, but some of them provide useful bug fixes and enhancement to the math commands and environments defined by \demopackage{amsmath}.

\begin{itemize}[noitemsep]
\item \demopackage{amsmath}
\item \demopackage{mathtools}
\item \demopackage{onlyamsmath}
\item \demopackage{braket}
\item \demopackage{cancel}
\item \demopackage{empheq}
\item \demopackage{exscale}
\item \demopackage{fixmath}
\item \demopackage{icomma}
\item \demopackage{xfrac}
\end{itemize}

\printCodeFromFile[125]{209}{preambel/packages.tex}

% ~~~~~~~~~~~~~~~~~~~~~~~~~~~~~~~~~~~~~~~~~~~~~~~~~~~~~~~~~~~~~~~~~~~~~~~~~
\subsection{Diagram and vector graphics packages}
Several approaches are possible to include vector graphics in a \latex document with \latex-code. In this template the packages \demopackage{tikz}/\demopackage{pgf} were chosen for this application.

Since \demopackage{tikz} and \demopackage{pgf} come with many options and extension package they are loaded in an extra file \file{preambel/packages-tikzpgf.tex}. The package \demopackage{pgfplots} provides an extension for scientific plots.

\begin{itemize}[noitemsep]
\item \demopackage{tikz}
\item \demopackage{pgf}
\item \demopackage{pgfplots}
\item \demopackage{pgfplotstable}
\end{itemize}


\printCodeFromFile[210]{229}{preambel/packages.tex}

% .........................................................................
\subsubsection{preambel/packages-tikzpgf.tex}
\labelfile{preambel/packages-tikzpgf.tex}

\printCodeFromFile[1]{92}{preambel/packages-tikzpgf.tex}

% ~~~~~~~~~~~~~~~~~~~~~~~~~~~~~~~~~~~~~~~~~~~~~~~~~~~~~~~~~~~~~~~~~~~~~~~~~
\subsection{Science packages}
Here packages are included which help to typeset numbers and units correctly.
The recommended package is \demopackage{siunitx}. The other packages are not activated by default because they are incompatible with \demopackage{siunitx} or not necessary with the default fonts.

\begin{itemize}[noitemsep]
\item \demopackage{siunitx}
\item not recommended: \demopackage{gensymb}, \demopackage{upgreek},  \demopackage{units}
\end{itemize}

\printCodeFromFile[230]{261}{preambel/packages.tex}

% ~~~~~~~~~~~~~~~~~~~~~~~~~~~~~~~~~~~~~~~~~~~~~~~~~~~~~~~~~~~~~~~~~~~~~~~~~
\subsection{Symbol packages}
There are many packages that provide additional symbols to \latex . Since these are font depended they are also incompatible if special font packages are loaded. Here only a selection of smybol packages is documented and loaded.

\begin{itemize}[noitemsep]
\item \demopackage{dsfont}
\item \demopackage{esint}
\item \demopackage{mathcomp}
\item \demopackage{euscript}
\item \demopackage{pifont}
\end{itemize}

\printCodeFromFile[263]{289}{preambel/packages.tex}

% ~~~~~~~~~~~~~~~~~~~~~~~~~~~~~~~~~~~~~~~~~~~~~~~~~~~~~~~~~~~~~~~~~~~~~~~~~
\subsection{Table packages}
Standard \latex tables are just ugly. In order to create good looking or even fancy tables further packages are necessary.

\begin{itemize}[noitemsep]
\item \demopackage{booktabs}
\item \demopackage{multirow}, \demopackage{bigstrut}
\item \demopackage{ltxtable}, \demopackage{tabularx}, \demopackage{longtable}
\item \demopackage{tabu}
\item \demopackage{tablestyles}
\end{itemize}
 
\printCodeFromFile[290]{325}{preambel/packages.tex}

% ~~~~~~~~~~~~~~~~~~~~~~~~~~~~~~~~~~~~~~~~~~~~~~~~~~~~~~~~~~~~~~~~~~~~~~~~~
\subsection{Text related packages}

This code is divided into bug fixing packages and packages for text-decoration, footnotes, references and lists.

\begin{itemize}[noitemsep]
\item \demopackage{ellipsis}
\item \demopackage{ulem}
\item \demopackage{soulutf8}
\item \demopackage{url}
\item \demopackage{footmisc}
\item (\demopackage{chngcntr})
\item (\demopackage{tablefootnote})
\item \demopackage{varioref}
\item \demopackage{cleveref}
\item \demopackage{enumitem}
\end{itemize}

\printCodeFromFile[326]{430}{preambel/packages.tex}

% ~~~~~~~~~~~~~~~~~~~~~~~~~~~~~~~~~~~~~~~~~~~~~~~~~~~~~~~~~~~~~~~~~~~~~~~~~
\subsection{Quotes}

The package \demopackage{csquotes} is a very powerful package that makes quotes language specific and in general easier.

\begin{itemize}[noitemsep]
\item \demopackage{csquotes}
\end{itemize}

\printCodeFromFile[332]{447}{preambel/packages.tex}

% ~~~~~~~~~~~~~~~~~~~~~~~~~~~~~~~~~~~~~~~~~~~~~~~~~~~~~~~~~~~~~~~~~~~~~~~~~
\subsection{Citation/bibliography packages}

There are many packages for citations and creation or modification or the bibliography. However almost all of them are nowadays replaced by the package 
\demopackage{biblatex} which provides the functionality of all previous package and beyond them. The enable the full functionality of \demopackage{biblatex} it is necessary to also replace \texttt{bibtex} by the program \texttt{biber}.

\begin{itemize}[noitemsep]
\item \demopackage{biblatex}
\end{itemize}

\printCodeFromFile[448]{480}{preambel/packages.tex}

% ~~~~~~~~~~~~~~~~~~~~~~~~~~~~~~~~~~~~~~~~~~~~~~~~~~~~~~~~~~~~~~~~~~~~~~~~~
\subsection{Packages for figures, placement and floats}
The basic package \demopackage{graphicx} for figures is already loaded at the beginning as shown in section \ref{sec:packages:base}. Here further packages are loaded that extent the placement and floating possibilities.

\begin{itemize}[noitemsep]
\item (\demopackage{float}  - replaced by \demopackage{floatrow})
\item \demopackage{wrapfig}
\item \demopackage{flafter}
\item \demopackage{placeins}
\item (\demopackage{floatflt}, unused alternative to \demopackage{wrapfig})
\end{itemize}

\printCodeFromFile[481]{519}{preambel/packages.tex}

% ~~~~~~~~~~~~~~~~~~~~~~~~~~~~~~~~~~~~~~~~~~~~~~~~~~~~~~~~~~~~~~~~~~~~~~~~~
\subsection{Caption packages}

The fundamental package for captions is the package \demopackage{caption}.  Its possibilities in terms of figure placement is enhanced by package \demopackage{floatrow} and for subfigures package \demopackage{subcaption}.

\begin{itemize}[noitemsep]
\item \demopackage{floatrow}, \demopackage{fr-fancy}
\item \demopackage{caption}
\item \demopackage{subcaption} (replaces \demopackage{subfig})
\item \demopackage{mcaption}
\item \demopackage{rotating}
\end{itemize}

\printCodeFromFile[521]{573}{preambel/packages.tex}

% ~~~~~~~~~~~~~~~~~~~~~~~~~~~~~~~~~~~~~~~~~~~~~~~~~~~~~~~~~~~~~~~~~~~~~~~~~
\subsection{Misc packages}
\label{sec:packages:misc}
This section contains mainly packages that should be loaded before \demopackage{hyperref} and do not fit into the other sections.
Currently it contains only the package \demopackage{lineno} for numbering lines
in the document. It is not loaded by default, but can be activated by removing the comment chars.

\begin{itemize}[noitemsep]
\item \demopackage{lineno} (unused)
\end{itemize}

\printCodeFromFile[576]{590}{preambel/packages.tex}

% ~~~~~~~~~~~~~~~~~~~~~~~~~~~~~~~~~~~~~~~~~~~~~~~~~~~~~~~~~~~~~~~~~~~~~~~~~
\subsection{Packages for index and other lists}
For the index package \demopackage{imakeidx} is loaded and for almost anything else \demopackage{glossaries} provides a solution.

\begin{itemize}[noitemsep]
\item \demopackage{imakeidx}
\item \demopackage{showidx}
\item \demopackage{glossaries}, \demopackage{glossary-longragged}
\end{itemize}

\printCodeFromFile[592]{630}{preambel/packages.tex}

% ~~~~~~~~~~~~~~~~~~~~~~~~~~~~~~~~~~~~~~~~~~~~~~~~~~~~~~~~~~~~~~~~~~~~~~~~~
\subsection{Verbatim packages}

Verbatim environments are used to display text in monospaced fonts. The typical usage is to display programming code. \demopackage{verbatim} and \demopackage{fancyvrb} are intended to be used for small (and fancy) code sections, whereas \demopackage{listings} is optimal for large code section with syntax highlighting.

The style of \demopackage{listings} is defined in file \file{preambel/style-listings.tex}.

\begin{itemize}[noitemsep]
\item \demopackage{upquote}
\item \demopackage{verbatim}
\item \demopackage{fancyvrb}
\item \demopackage{listings}
\end{itemize}

\printCodeFromFile[633]{660}{preambel/packages.tex}

% ~~~~~~~~~~~~~~~~~~~~~~~~~~~~~~~~~~~~~~~~~~~~~~~~~~~~~~~~~~~~~~~~~~~~~~~~~
\subsection{Fancy packages}

Two different types of fancy packages are loaded. \demopackage{lettrine} for dropping capitals and other packages for fancy framed texts: \demopackage{boxedminipage}, \demopackage{fancybox}, \demopackage{framed} and \demopackage{mdframed}. Not however that \demopackage{mdframed} is a modern package that can replace the other three.

\begin{itemize}[noitemsep]
\item \demopackage{lettrine} 
\item \demopackage{boxedminipage}
\item \demopackage{framed}
\item \demopackage{fancybox} (incompatible with fancyvrb)
\item \demopackage{mdframed}
\end{itemize}

\printCodeFromFile[661]{695}{preambel/packages.tex}

% ~~~~~~~~~~~~~~~~~~~~~~~~~~~~~~~~~~~~~~~~~~~~~~~~~~~~~~~~~~~~~~~~~~~~~~~~~
\subsection{Layout packages}

The indentation of the first paragraph can be modified using \demopackage{indentation}. The text can be printed in multiple columns with package \demopackage{multicol}. The line spacing can be modified using package 
\demopackage{setspace}. And the page layout can be modified with the packages \demopackage{geometry} or alternatively \demopackage{typearea}. The latter is automatically loaded with the koma script class. \demopackage{changepage} can be used to detect odd/even pages.

The configuration of most packages is in file \file{preambel/style.tex} and for package \demopackage{geometry} in file \file{preambel/style-geometry.tex}.


\begin{itemize}[noitemsep]
\item \demopackage{indentation} (unused)
\item \demopackage{multicol}
\item \demopackage{setspace}
\item \demopackage{geometry} (unused)
\item \demopackage{typearea} (automatically loaded)
\item \demopackage{geometry} (changepage)
\end{itemize}

\printCodeFromFile[697]{739}{preambel/packages.tex}

% ~~~~~~~~~~~~~~~~~~~~~~~~~~~~~~~~~~~~~~~~~~~~~~~~~~~~~~~~~~~~~~~~~~~~~~~~~
\subsection{Packages for header and footer}

The content in the header and footer of a page is defined with package \demopackage{scrpage2}, with the settings defined in file \file{preambel/style-scrpage2.tex}.

The total number of page is provided by package \demopackage{pageslts}.

\begin{itemize}[noitemsep]
\item \demopackage{scrpage2}
\item \demopackage{pageslts}
\end{itemize}

\printCodeFromFile[740]{778}{preambel/packages.tex}

% ~~~~~~~~~~~~~~~~~~~~~~~~~~~~~~~~~~~~~~~~~~~~~~~~~~~~~~~~~~~~~~~~~~~~~~~~~
\subsection{Layout of headings}

All headings can be redefined using package \demopackage{titlesec}.

\printCodeFromFile[779]{793}{preambel/packages.tex}

% ~~~~~~~~~~~~~~~~~~~~~~~~~~~~~~~~~~~~~~~~~~~~~~~~~~~~~~~~~~~~~~~~~~~~~~~~~
\subsection{Layout of table of contents}

The format of the table of contents and other lists is defined by package \demopackage{tocstyle}. The appendix title can be modified with package \demopackage{appendix}

\begin{itemize}[noitemsep]
\item \demopackage{tocstyle}
\item \demopackage{appendix} (unused)
\end{itemize}

\printCodeFromFile[780]{837}{preambel/packages.tex}

% ~~~~~~~~~~~~~~~~~~~~~~~~~~~~~~~~~~~~~~~~~~~~~~~~~~~~~~~~~~~~~~~~~~~~~~~~~
\subsection{PDF packages (including hyperref)}

\demopackage{pdfpages} is a package for the inclusion of pages from external pdf documents,
\demopackage{pdflscape} for changing the page orientation,
\demopackage{microtype} for improving the textformating,
\demopackage{hyperref} for almost everything else that is related to PDF especially its hyperlinks and 
\demopackage{bookmark} for bookmarks in a PDF document.

Note that \demopackage{hyperref} must be loaded after almost all packages!

The settings of hyperref are defined in file \file{preambel/style-hyperref.tex}.

\begin{itemize}[noitemsep]
\item \demopackage{pdfpages}
\item \demopackage{pdflscape} (unused)
\item \demopackage{microtype}
\item \demopackage{hyperref}
\item \demopackage{bookmark}
\end{itemize}

\printCodeFromFile[838]{895}{preambel/packages.tex}

% ~~~~~~~~~~~~~~~~~~~~~~~~~~~~~~~~~~~~~~~~~~~~~~~~~~~~~~~~~~~~~~~~~~~~~~~~~
\subsection{Additional packages  (explicitly after package hyperref)}

These packages here have nothing in common except that they can be loaded after \demopackage{hyperref}. Other additional package that must be loaded before must be put into the section \texttt{Misc Packages}, see section \cref{sec:packages:misc}.

\printCodeFromFile[896]{918}{preambel/packages.tex}

% -------------------------------------------------------------------------
\section{preambel/style.tex}
\labelfile{preambel/style.tex}

% ~~~~~~~~~~~~~~~~~~~~~~~~~~~~~~~~~~~~~~~~~~~~~~~~~~~~~~~~~~~~~~~~~~~~~~~~~
\subsection{Package sections}

This is the file that defines all settings for the package including the page layout. The settings are grouped together according to there usage. 
These section defined at the beginning of the file:

\printCodeFromFile[1]{27}{preambel/style.tex}

If you do not require all sections in your document you can change the setting from \emph{true} to \emph{false} in all section definitions you do not want to include in the compilation.

% ~~~~~~~~~~~~~~~~~~~~~~~~~~~~~~~~~~~~~~~~~~~~~~~~~~~~~~~~~~~~~~~~~~~~~~~~~
\subsection{Colors}

If package \demopackage{xcolor} is loaded then colors for the sections, the tables and pdf links are defined with \democs{definecolor} and \democs{colorlet}. Note that \democs{SetTemplateDefinition} is used here to define switchable colors for different document targets (web/print).

\printCodeFromFile[28]{69}{preambel/style.tex}

% ~~~~~~~~~~~~~~~~~~~~~~~~~~~~~~~~~~~~~~~~~~~~~~~~~~~~~~~~~~~~~~~~~~~~~~~~~
\subsection{Math}

This code shows how to exchange the vector symbol arrow with a bold font and how to exchange various greek symbols by there \emph{var} variant.

\printCodeFromFile[70]{96}{preambel/style.tex}

% ~~~~~~~~~~~~~~~~~~~~~~~~~~~~~~~~~~~~~~~~~~~~~~~~~~~~~~~~~~~~~~~~~~~~~~~~~
\subsection{Science}

Loading of \file{preambel/style-siunitx.tex}.

\printCodeFromFile[97]{105}{preambel/style.tex}

% .........................................................................
\subsubsection{preambel/style-siunitx.tex}
\labelfile{preambel/style-siunitx.tex}

\demopackage{siunitx} is setup for the detection of all font changes and in mode \emph{math}. For german text several changes are applied to ensure the correct setting of math in that language.

Additionally the commands \democs{nicefrac}, \democs{unitfrac} and \democs{unit} are defined in order to emulate the commands from the package \demopackage{units}.

\printCodeFromFile[1]{36}{preambel/style-siunitx.tex}

% ~~~~~~~~~~~~~~~~~~~~~~~~~~~~~~~~~~~~~~~~~~~~~~~~~~~~~~~~~~~~~~~~~~~~~~~~~
\subsection{Diagrams}

Setup of default plot size for tikz/pgfplots and in case of german text the decimal separator is set up as a comma.

Further settings for pgfplots are in a separate file: \file{preambel/style-pgfplots.tex}.

\printCodeFromFile[106]{126}{preambel/style.tex}

% .........................................................................
\subsubsection{preambel/style-pgfplots.tex}
\labelfile{preambel/style-pgfplots.tex}

Color series for pgfplots are defined in this file.

\printCodeFromFile[1]{31}{preambel/style-pgfplots.tex}

% ~~~~~~~~~~~~~~~~~~~~~~~~~~~~~~~~~~~~~~~~~~~~~~~~~~~~~~~~~~~~~~~~~~~~~~~~~
\subsection{Text}

Here the font for urls (package \demopackage{url}) and the font in margins used by package \demopackage{marginnote} is defined.

\printCodeFromFile[127]{145}{preambel/style.tex}

% ~~~~~~~~~~~~~~~~~~~~~~~~~~~~~~~~~~~~~~~~~~~~~~~~~~~~~~~~~~~~~~~~~~~~~~~~~
\subsection{Footnotes}

\printCodeFromFile[146]{186}{preambel/style.tex}

% =========================================================================
\chapter{How to list}

% =========================================================================
\chapter{Known problems}

% -------------------------------------------------------------------------
\section{No room for new write}
\label{sec:problems:write}