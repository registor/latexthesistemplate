\chapter*{Curriculum Vitae}
\markboth{Curriculum Vitae}{Curriculum Vitae}

\IfPackageLoaded{hyperref}{
	\phantomsection
	\addcontentsline{toc}{chapter}{Curriculum Vitae}
}

\IfPackageLoaded{currvita}{%

% Remeber that you do not write this CV to apply for a job.
% This is just a brief summery of you previous research career.
% A 'real' CV is much more complex!

% This is the CV that I submitted in my phd thesis in summer of 2011.
% It is in german. I have not translated it into english, because
% I am not familier with english CV styles.

\begin{cv}{}
\begin{cvlist}{Personalien}
	\item[Name]
		Matthias Pospiech \\
		geboren am 11.06.79 in Herdecke \\
		verheiratet, deutsch
\end{cvlist}
%
\begin{cvlist}{Schulbildung}
	\item[1998] Abitur, Gymnasium Petrinum in Brilon 
\end{cvlist}
%
\begin{cvlist}{Zivildienst}
	\item[07/98 - 08/99] 
	Pädagogische Betreuung am Internat Collegium Aloysianum, Werl
\end{cvlist}
%
\begin{cvlist}{Studium}
	\item[WS/99 - SS/02] TU Kaiserslautern, Studium der technischen Physik
	\item[09/02 - 02/03] Auslandsaufenthalt an der University of Sheffield (UK)
	\item[SS/03 - SS/06] Universität Hannover, Studium der technischen Physik
	\\[0.5\baselineskip]
Thema der Diplomarbeit: \enquote{Charakterisierung des Rauschverhaltens eines 
weit abstimmbaren Ytterbium dotierten kerngepumpten Faserlasers}, durchgeführt 
am Laserzentrum Hannover e.\,V.
	\item[Mai 2006] Abschluss: Diplom-Physiker	
\end{cvlist}
%
\begin{cvlist}{Promotion}
	\item[09/2006 - heute] Wissenschaftlicher Mitarbeiter am Institut für 
Quantenoptik, Leibniz Universität Hannover
\end{cvlist}

\end{cv}

}%